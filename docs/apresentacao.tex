\documentclass[aspectratio=169]{beamer}
\usepackage[utf8]{inputenc}
\usepackage[brazil]{babel}
\usepackage{graphicx}
\usepackage{listings}
\usepackage{xcolor}
\usepackage{tikz}
\usetikzlibrary{positioning}

% Tema
\usetheme{Madrid}
\usecolortheme{default}

% Configuração de cores
\definecolor{ciscoBlue}{RGB}{0,51,153}
\definecolor{codeBackground}{RGB}{245,245,245}

\setbeamercolor{structure}{fg=ciscoBlue}
\setbeamercolor{frametitle}{fg=white,bg=ciscoBlue}

% Configuração de código
\lstset{
    basicstyle=\ttfamily\small,
    backgroundcolor=\color{codeBackground},
    frame=single,
    breaklines=true,
    numbers=left,
    numberstyle=\tiny,
    keywordstyle=\color{blue},
    commentstyle=\color{green!50!black},
    stringstyle=\color{red}
}

% Informações do documento
\title[Servidor de Processamento]{Sistema Cliente-Servidor de Processamento Distribuído}
\subtitle{Comunicação TCP/UDP com Análise de Rede}
\author{Giuseppe Sena}
\institute{Redes de Computadores}
\date{\today}

\begin{document}

% Slide de título
\begin{frame}
    \titlepage
\end{frame}

% Sumário
\begin{frame}{Sumário}
    \tableofcontents
\end{frame}

% ============================================
% SEÇÃO 1: INTRODUÇÃO
% ============================================
\section{Introdução}

\begin{frame}{Visão Geral do Projeto}
    \begin{columns}
        \begin{column}{0.5\textwidth}
            \textbf{Objetivo:}
            \begin{itemize}
                \item Sistema cliente-servidor
                \item Processamento distribuído
                \item Comunicação TCP e UDP
                \item Análise de tráfego de rede
            \end{itemize}
        \end{column}
        \begin{column}{0.5\textwidth}
            \textbf{Tecnologias:}
            \begin{itemize}
                \item Java (Sockets)
                \item Cisco Packet Tracer
                \item Wireshark
                \item Protocolos TCP/UDP
            \end{itemize}
        \end{column}
    \end{columns}

    \vspace{1cm}
    \begin{alertblock}{Funcionalidades Implementadas}
        5 tipos de tarefas: Ordenação, Busca, Multiplicação de Matrizes, Números Primos e Transferência de Arquivos
    \end{alertblock}
\end{frame}

\begin{frame}{Arquitetura do Sistema}
    \begin{center}
        \begin{tikzpicture}[scale=0.85]
            % Cliente
            \node[draw, rectangle, minimum width=2.5cm, minimum height=1.5cm, fill=blue!20] (cliente) at (0,0) {\textbf{Cliente}};
            \node[below=0.1cm of cliente] {\small Interface Swing};

            % Servidor
            \node[draw, rectangle, minimum width=2.5cm, minimum height=1.5cm, fill=green!20] (servidor) at (8,0) {\textbf{Servidor}};
            \node[below=0.1cm of servidor] {\small Processamento};

            % Setas TCP
            \draw[->, thick, red] (cliente.north) .. controls (4,2) .. (servidor.north) node[midway, above] {TCP 5000 - Tarefas};

            % Setas UDP
            \draw[->, thick, blue] (servidor.south) .. controls (4,-2) .. (cliente.south) node[midway, below] {UDP 5001 - Status};
        \end{tikzpicture}
    \end{center}

    \vspace{0.5cm}
    \begin{block}{Protocolos}
        \textbf{TCP (5000):} Envio de tarefas e recebimento de resultados \\
        \textbf{UDP (5001):} Monitoramento em tempo real (CPU, memória, threads)
    \end{block}
\end{frame}

% ============================================
% SEÇÃO 2: TOPOLOGIA
% ============================================
\section{Topologia de Rede}

\begin{frame}{Topologia no Cisco Packet Tracer}
    \begin{center}
        \begin{tikzpicture}[scale=0.9, every node/.style={scale=0.9}]
            % PC1
            \node[draw, rectangle, minimum width=1.8cm, minimum height=1cm, fill=yellow!30] (pc1) at (0,3) {PC1};
            \node[below=0.1cm of pc1, font=\tiny] {Cliente};
            \node[below=0.35cm of pc1, font=\tiny] {192.168.1.10};

            % Roteadores
            \node[draw, circle, minimum size=1.2cm, fill=cyan!30] (r1) at (3,3) {R1};
            \node[draw, circle, minimum size=1.2cm, fill=cyan!30] (r2) at (6,3) {R2};
            \node[draw, circle, minimum size=1.2cm, fill=cyan!30] (r3) at (9,3) {R3};

            % PC2
            \node[draw, rectangle, minimum width=1.8cm, minimum height=1cm, fill=yellow!30] (pc2) at (12,3) {PC2};
            \node[below=0.1cm of pc2, font=\tiny] {Servidor};
            \node[below=0.35cm of pc2, font=\tiny] {10.0.0.100};

            % Conexões
            \draw[-] (pc1) -- (r1);
            \draw[-] (r1) -- (r2);
            \draw[-] (r2) -- (r3);
            \draw[-] (r3) -- (pc2);
        \end{tikzpicture}
    \end{center}

    \begin{block}{Endereçamento}
        \begin{itemize}
            \item \textbf{PC1:} 192.168.1.10/16
            \item \textbf{PC2:} 10.0.0.100/8
            \item Roteadores com NAT para interligação das redes
        \end{itemize}
    \end{block}
\end{frame}

% ============================================
% SEÇÃO 3 — JAVA
% ============================================
\section{Aplicação Java}

\begin{frame}{Estrutura do Código}
    \begin{columns}
        \begin{column}{0.5\textwidth}
            \textbf{Classes Principais:}
            \begin{itemize}
                \item \texttt{Main.java} - Menu principal
                \item \texttt{FileServer.java} - Servidor TCP/UDP
                \item \texttt{FileClient.java} - Cliente TCP/UDP
                \item \texttt{ProcessingServer.java} - Processamento
                \item \texttt{ProcessingClient.java} - Requisições
                \item \texttt{Config.java} - Configurações
                \item \texttt{ConfigDialog.java} - Interface
            \end{itemize}
        \end{column}
        \begin{column}{0.5\textwidth}
            \textbf{Tarefas Implementadas:}
            \begin{enumerate}
                \item \textbf{SORT} - Ordenação de arrays
                \item \textbf{SEARCH} - Busca binária
                \item \textbf{MATRIX} - Multiplicação de matrizes
                \item \textbf{PRIME} - Números primos
                \item \textbf{FILE} - Transferência de arquivos
            \end{enumerate}
        \end{column}
    \end{columns}
    
    \vspace{0.5cm}
    \begin{block}{Características}
        Interface gráfica Swing, processamento multithreaded, monitoramento em tempo real
    \end{block}
\end{frame}

\begin{frame}[fragile]{Servidor - Código Principal}
\begin{lstlisting}[language=Java, basicstyle=\ttfamily\tiny]
public class FileServer {
    private ServerSocket tcpServer;
    private DatagramSocket udpServer;
    private volatile boolean running;

    private void startTCPServer() throws IOException {
        tcpServer = new ServerSocket(5000);
        while (running) {
            Socket clientSocket = tcpServer.accept();
            new Thread(() -> handleClient(clientSocket)).start();
        }
    }

    private void startUDPServer() throws IOException {
        udpServer = new DatagramSocket(5001);
        while (running) {
            byte[] buffer = new byte[1024];
            DatagramPacket packet = new DatagramPacket(buffer, buffer.length);
            udpServer.receive(packet);

            String metrics = getServerMetrics();
            byte[] response = metrics.getBytes();
            DatagramPacket responsePacket =
                new DatagramPacket(response, response.length,
                packet.getAddress(), packet.getPort());
            udpServer.send(responsePacket);
        }
    }
}
\end{lstlisting}
\end{frame}

\begin{frame}[fragile]{Cliente - Código Principal}
    \begin{lstlisting}[language=Java, basicstyle=\ttfamily\tiny]
public class FileClient {
    private Socket tcpSocket;
    private DatagramSocket udpSocket;
    
    // Conectar ao servidor via TCP
    public void connect(String host, int port) throws IOException {
        tcpSocket = new Socket(host, port);
        appendLog("Conectado ao servidor: " + host + ":" + port);
    }
    
    // Enviar tarefa via TCP
    public String sendTask(String taskType, String params) throws IOException {
        PrintWriter out = new PrintWriter(tcpSocket.getOutputStream(), true);
        BufferedReader in = new BufferedReader(
            new InputStreamReader(tcpSocket.getInputStream())
        );
        
        // Enviar requisicao
        out.println(taskType + "|" + params);
        
        // Receber resposta
        return in.readLine();
    }
    
    // Monitorar status via UDP
    private void startStatusMonitoring() {
        new Thread(() -> {
% ============================================
% SEÇÃO 4 — WIRESHARK
% ============================================
\section{Análise no Wireshark}

\begin{frame}{Captura de Tráfego}
    \textbf{Configuração da Captura:}
    \begin{itemize}
        \item \textbf{Interface:} Loopback (lo0 no macOS)
        \item \textbf{Filtros aplicados:}
            \begin{itemize}
                \item \texttt{tcp.port == 5000} - Tráfego TCP
                \item \texttt{udp.port == 5001} - Tráfego UDP
                \item \texttt{tcp.port == 5000 || udp.port == 5001} - Ambos
            \end{itemize}
    \end{itemize}
    
    \vspace{0.5cm}
    
    \begin{block}{Procedimento}
        \begin{enumerate}
            \item Iniciar captura no Wireshark
            \item Executar servidor e cliente
            \item Realizar uma tarefa
            \item Parar captura e aplicar filtros
            \item Analisar pacotes TCP e UDP
        \end{enumerate}
    \end{block}
\end{frame}

\begin{frame}{Análise TCP - Three-Way Handshake}
    \begin{center}
        \begin{tikzpicture}[scale=0.85]
            \node at (0,4) {\textbf{Cliente}};
            \node at (8,4) {\textbf{Servidor}};

            \draw[thick] (0,3.5) -- (0,0);
            \draw[thick] (8,3.5) -- (8,0);

            \draw[->, thick, red] (0,3) -- (8,2.5) node[midway, above] {SYN};
            \draw[->, thick, blue] (8,2) -- (0,1.5) node[midway, above] {SYN-ACK};
            \draw[->, thick, green!50!black] (0,1) -- (8,0.5) node[midway, above] {ACK};

            \node at (4,0) {\textcolor{green}{Conexão Estabelecida}};
        \end{tikzpicture}
    \end{center}
    
    \begin{alertblock}{Observação}
        O handshake TCP garante que ambas as partes estão prontas para comunicação confiável
    \end{alertblock}
\end{frame}

\begin{frame}{Análise TCP - Transferência de Dados}
    \textbf{Características Observadas:}
    
    \begin{columns}
        \begin{column}{0.5\textwidth}
            \textbf{Pacotes de Dados:}
            \begin{itemize}
                \item Flags: PSH, ACK
                \item Sequence numbers incrementais
                \item ACK após cada pacote
                \item Controle de fluxo (Window Size)
                \item Retransmissão em caso de perda
            \end{itemize}
        \end{column}
        \begin{column}{0.5\textwidth}
            \textbf{Overhead TCP:}
            \begin{itemize}
                \item Header: 20-60 bytes
                \item ACKs: packets adicionais
                \item Handshake: 3 pacotes
                \item Finalização: 4 pacotes (FIN)
                \item Total: \textasciitilde40-50\% overhead
            \end{itemize}
        \end{column}
    \end{columns}
    
    \vspace{0.5cm}
    \begin{exampleblock}{Vantagem}
        Entrega confiável e ordenada dos dados - ideal para tarefas críticas
    \end{exampleblock}
\end{frame}

\begin{frame}[fragile]{Análise UDP - Monitoramento}
    \textbf{Características Observadas:}
    
    \begin{block}{Tráfego UDP}
        \begin{itemize}
            \item \textbf{Sem handshake:} Comunicação direta
            \item \textbf{Sem ACKs:} Não há confirmação de recebimento
            \item \textbf{Overhead mínimo:} Header de apenas 8 bytes
            \item \textbf{Periódico:} Requisições a cada 2 segundos
            \item \textbf{Payload JSON:} Métricas do servidor
        \end{itemize}
    \end{block}
    
    \vspace{0.3cm}
    
    \textbf{Exemplo de Payload (JSON):}
    \begin{lstlisting}[language=json, basicstyle=\ttfamily\tiny]
{
  "cpu": "15.2%",
  "memoria": "512MB/2048MB",
  "threads": 12,
  "tarefas_processadas": 156,
  "tempo_medio": "342ms"
}
    \end{lstlisting}
    
    \begin{exampleblock}{Vantagem}
        Baixo overhead - ideal para monitoramento em tempo real
    \end{exampleblock}
\end{frame}

\begin{frame}{Comparação: TCP vs UDP}
    \begin{center}
        \small
        \begin{tabular}{|l|c|c|}
            \hline
            \textbf{Característica} & \textbf{TCP} & \textbf{UDP} \\
            \hline\hline
            Handshake & 3 pacotes & 0 pacotes \\
            \hline
            Header & 20-60 bytes & 8 bytes \\
            \hline
            Confirmação (ACK) & Sim & Não \\
            \hline
            Ordenação & Garantida & Não garantida \\
            \hline
            Retransmissão & Sim & Não \\
            \hline
            Controle de fluxo & Sim & Não \\
            \hline
            Overhead & Alto (\textasciitilde40-50\%) & Baixo (\textasciitilde5\%) \\
            \hline
            Velocidade & Mais lento & Mais rápido \\
            \hline
            Confiabilidade & Alta & Baixa \\
            \hline
            \textbf{Uso no projeto} & Tarefas críticas & Status/Monitor \\
            \hline
        \end{tabular}
    \end{center}
    
    \vspace{0.5cm}
    \begin{alertblock}{Decisão de Design}
        TCP para dados críticos (tarefas e resultados) \\
        UDP para informações não-críticas (métricas em tempo real)
    \end{alertblock}
\end{frame}

% ============================================
% SEÇÃO 5: RESULTADOS
% ============================================
\section{Resultados e Conclusões}

\begin{frame}{Resultados Obtidos}
    \begin{block}{Funcionalidades Validadas}
        \begin{itemize}
            \item[$\checkmark$] Sistema cliente-servidor totalmente funcional
            \item[$\checkmark$] Comunicação TCP confiável para tarefas
            \item[$\checkmark$] Monitoramento UDP em tempo real
            \item[$\checkmark$] Roteamento multi-hop (3 roteadores)
            \item[$\checkmark$] NAT e port forwarding funcionando
            \item[$\checkmark$] Interface gráfica intuitiva
            \item[$\checkmark$] 5 tipos de tarefas implementadas
        \end{itemize}
    \end{block}
    
    \vspace{0.3cm}
    
    \begin{exampleblock}{Métricas de Performance}
        \begin{itemize}
            \item Tempo de resposta médio: 200-500ms
            \item Taxa de perda de pacotes UDP: < 0.1\%
            \item Throughput TCP: \textasciitilde10 MB/s (localhost)
            \item Overhead de rede: TCP 45\%, UDP 5\%
        \end{itemize}
    \end{exampleblock}
\end{frame}

\begin{frame}{Desafios Encontrados}
    \begin{alertblock}{Desafios Técnicos}
        \begin{enumerate}
            \item \textbf{Configuração NAT:}
                \begin{itemize}
                    \item Port forwarding para múltiplas portas
                    \item Tradução de endereços entre redes
                \end{itemize}
            
            \item \textbf{Sincronização TCP/UDP:}
                \begin{itemize}
                    \item Garantir que ambos os servidores iniciem corretamente
                    \item Tratar desconexões e reconexões
                \end{itemize}
            
            \item \textbf{Interface Gráfica:}
                \begin{itemize}
                    \item Thread-safety no Swing
                    \item Atualização em tempo real sem travar a UI
                \end{itemize}
            
            \item \textbf{Análise no Wireshark:}
                \begin{itemize}
                    \item Filtrar tráfego relevante
                    \item Interpretar payloads binários
                \end{itemize}
        \end{enumerate}
    \end{alertblock}
\end{frame}

\begin{frame}{Aprendizados}
    \textbf{Conceitos de Redes Aplicados:}
    
    \begin{columns}
        \begin{column}{0.5\textwidth}
            \textbf{Camada de Transporte:}
            \begin{itemize}
                \item TCP three-way handshake
                \item Sequence/ACK numbers
                \item Controle de fluxo
                \item Retransmissão
                \item UDP connectionless
            \end{itemize}
        \end{column}
        \begin{column}{0.5\textwidth}
            \textbf{Camada de Rede:}
            \begin{itemize}
                \item Roteamento estático
                \item NAT e port forwarding
                \item Sub-redes e endereçamento
                \item Traceroute
                \item ARP
            \end{itemize}
        \end{column}
    \end{columns}
    
    \vspace{0.5cm}
    
    \begin{block}{Habilidades Desenvolvidas}
        \begin{itemize}
            \item Programação de sockets em Java
            \item Configuração de equipamentos Cisco
            \item Análise de tráfego com Wireshark
            \item Design de protocolos de aplicação
        \end{itemize}
    \end{block}
\end{frame}

\begin{frame}{Possíveis Melhorias}
    \textbf{Melhorias Futuras:}
    
    \begin{enumerate}
        \item \textbf{Segurança:}
            \begin{itemize}
                \item Implementar TLS/SSL para TCP
                \item Autenticação de clientes
                \item Criptografia de payloads
            \end{itemize}
        
        \item \textbf{Escalabilidade:}
            \begin{itemize}
                \item Pool de threads no servidor
                \item Balanceamento de carga
                \item Cache de resultados
            \end{itemize}
        
        \item \textbf{Funcionalidades:}
            \begin{itemize}
                \item Mais tipos de tarefas
                \item Enfileiramento de tarefas
                \item Priorização de requisições
            \end{itemize}
        
        \item \textbf{Monitoramento:}
            \begin{itemize}
                \item Logs detalhados
                \item Métricas adicionais
                \item Dashboard web
            \end{itemize}
    \end{enumerate}
\end{frame}

\begin{frame}{Conclusão}
    \begin{block}{Objetivos Alcançados}
        \begin{itemize}
            \item[$\checkmark$] Sistema funcional de processamento distribuído
            \item[$\checkmark$] Comunicação eficiente TCP/UDP
            \item[$\checkmark$] Roteamento multi-hop configurado
            \item[$\checkmark$] Análise detalhada do tráfego de rede
            \item[$\checkmark$] Documentação completa
        \end{itemize}
    \end{block}
    
    \vspace{0.5cm}
    
    \begin{exampleblock}{Considerações Finais}
        O projeto demonstrou com sucesso a aplicação prática de conceitos de redes de computadores, integrando programação, configuração de equipamentos e análise de tráfego. A escolha adequada dos protocolos (TCP para confiabilidade, UDP para eficiência) resultou em um sistema robusto e eficiente.
    \end{exampleblock}
\end{frame}

% ============================================
% SLIDE FINAL
% ============================================
\begin{frame}[plain]
    \begin{center}
        \Huge \textcolor{ciscoBlue}{\textbf{Obrigado!}}
        
        \vspace{1cm}
        
        \Large Perguntas?
        
        \vspace{1cm}
        
        \normalsize
        \textbf{Giuseppe Sena} \\
        Redes de Computadores \\
        \today
        
        \vspace{0.5cm}
        
        \small
        \textit{Código-fonte disponível em:} \\
        \texttt{github.com/giusfds/serverShow}
    \end{center}
\end{frame}

\end{document}  \begin{itemize}
                    \item Ordenação de 10.000 elementos
                    \item Busca em array ordenado
                    \item Multiplicação de matrizes 100x100
                    \item Cálculo de números primos até 100.000
                    \item Transferência de arquivo
                \end{itemize}
            \item Monitoramento em tempo real via UDP
        \end{enumerate}
    \end{block}
\end{frame}
\begin{frame}{Testes de Conectividade}
    \textbf{Comandos executados no PC1:}
    
    \begin{block}{Teste de Ping}
        \texttt{PC1> ping 10.0.0.100} \\
        \textcolor{green}{Reply from 10.0.0.100: bytes=32 time<1ms TTL=125}
    \end{block}
    
    \begin{block}{Traceroute}
        \texttt{PC1> tracert 10.0.0.100}
        \begin{itemize}
            \item Hop 1: 192.168.0.1 (R1)
            \item Hop 2: 172.16.0.1 (R2)
            \item Hop 3: 10.0.0.1 (R3)
            \item Hop 4: 10.0.0.100 (PC2)
        \end{itemize}
    \end{block}
    
    \begin{exampleblock}{Resultado}
        Rede totalmente funcional com 3 saltos entre cliente e servidor
    \end{exampleblock}
\end{frame}

% ============================================
% SEÇÃO 3 — JAVA
% ============================================
\section{Aplicação Java}

\begin{frame}[fragile]{Servidor - Código Principal}
\begin{lstlisting}[language=Java, basicstyle=\ttfamily\tiny]
public class FileServer {
    private ServerSocket tcpServer;
    private DatagramSocket udpServer;
    private volatile boolean running;

    private void startTCPServer() throws IOException {
        tcpServer = new ServerSocket(5000);
        while (running) {
            Socket clientSocket = tcpServer.accept();
            new Thread(() -> handleClient(clientSocket)).start();
        }
    }

    private void startUDPServer() throws IOException {
        udpServer = new DatagramSocket(5001);
        while (running) {
            byte[] buffer = new byte[1024];
            DatagramPacket packet = new DatagramPacket(buffer, buffer.length);
            udpServer.receive(packet);

            String metrics = getServerMetrics();
            byte[] response = metrics.getBytes();
            DatagramPacket responsePacket =
                new DatagramPacket(response, response.length,
                packet.getAddress(), packet.getPort());
            udpServer.send(responsePacket);
        }
    }
}
\end{lstlisting}
\end{frame}

% ============================================
% SEÇÃO 4 — WIRESHARK
% ============================================
\section{Análise no Wireshark}

\begin{frame}{Análise TCP - Three-Way Handshake}
    \begin{center}
        \begin{tikzpicture}[scale=0.85]
            \node at (0,4) {\textbf{Cliente}};
            \node at (8,4) {\textbf{Servidor}};

            \draw[thick] (0,3.5) -- (0,0);
            \draw[thick] (8,3.5) -- (8,0);

            \draw[->, thick, red] (0,3) -- (8,2.5) node[midway, above] {SYN};
            \draw[->, thick, blue] (8,2) -- (0,1.5) node[midway, above] {SYN-ACK};
            \draw[->, thick, green!50!black] (0,1) -- (8,0.5) node[midway, above] {ACK};

            \node at (4,0) {\textcolor{green}{Conexão Estabelecida}};
        \end{tikzpicture}
    \end{center}
\end{frame}

\end{document}
