\documentclass[12pt,a4paper]{article}

% Pacotes essenciais
\usepackage[utf8]{inputenc}
\usepackage[portuguese]{babel}
\usepackage[T1]{fontenc}
\usepackage{graphicx}
\usepackage{float}
\usepackage{amsmath}
\usepackage{amssymb}
\usepackage{hyperref}
\usepackage{listings}
\usepackage{xcolor}
\usepackage{booktabs}
\usepackage{geometry}
\usepackage{enumitem}
\usepackage{tocloft}
\usepackage{fancyhdr}

% Configuração de geometria
\geometry{
    left=3cm,
    right=2cm,
    top=3cm,
    bottom=2cm
}

% Configuração de cabeçalho e rodapé
\pagestyle{fancy}
\fancyhf{}
\rhead{\leftmark}
\lhead{Sistema de Processamento Distribuído}
\rfoot{\thepage}

% Configuração de hyperlinks
\hypersetup{
    colorlinks=true,
    linkcolor=blue,
    filecolor=magenta,
    urlcolor=cyan,
    citecolor=blue,
}

% Configuração de código
\lstdefinestyle{mystyle}{
    backgroundcolor=\color{gray!10},
    commentstyle=\color{green!50!black},
    keywordstyle=\color{blue},
    numberstyle=\tiny\color{gray},
    stringstyle=\color{orange},
    basicstyle=\ttfamily\footnotesize,
    breakatwhitespace=false,
    breaklines=true,
    captionpos=b,
    keepspaces=true,
    numbers=left,
    numbersep=5pt,
    showspaces=false,
    showstringspaces=false,
    showtabs=false,
    tabsize=2,
    frame=single
}
\lstset{style=mystyle}

% Comando para checkbox
\newcommand{\checkbox}{\rlap{$\square$}{\raisebox{2pt}{\large\hspace{1pt}\checkmark}}}

\begin{document}

% Página de título
\begin{titlepage}
    \centering
    \vspace*{2cm}
    
    {\Huge\bfseries RELATÓRIO TÉCNICO\par}
    \vspace{1cm}
    {\LARGE Sistema de Processamento Distribuído com TCP/UDP\par}
    \vspace{2cm}
    
    {\Large\textbf{Informações do Projeto}\par}
    \vspace{0.5cm}
    
    {\large
    \textbf{Disciplina:} Redes de Computadores\par
    \textbf{Aluno(s):} André Luis Silva, Caio Diniz, Giuseppe Cordeiro, Vinícius Miranda\par
    \textbf{Professor:} Max Machado\par
    }
    
    \vfill
    
    {\large \today\par}
\end{titlepage}

% Sumário
\tableofcontents
\newpage

% Lista de figuras
\listoffigures
\newpage

% Lista de tabelas
\listoftables
\newpage

\section{INTRODUÇÃO}

\subsection{Contexto}

Este projeto implementa um \textbf{Sistema de Processamento Distribuído} que simula um ambiente de análise de performance de algoritmos computacionais. O sistema permite que um cliente envie tarefas computacionalmente intensivas para um servidor remoto processar, retornando os resultados e métricas de execução.

O contexto escolhido combina conceitos de múltiplas disciplinas:

\begin{itemize}
    \item \textbf{Redes de Computadores}: Comunicação TCP/UDP, roteamento, NAT
    \item \textbf{Arquitetura de Computadores}: Processamento paralelo, otimização
    \item \textbf{Sistemas Operacionais}: Multithreading, sincronização, pools de threads
    \item \textbf{Análise de Algoritmos}: Comparação de complexidade e performance
\end{itemize}

\subsection{Objetivos}

\textbf{Objetivo Geral:}

Desenvolver um sistema cliente-servidor completo que demonstre o uso de protocolos TCP e UDP, multithreading, e processamento distribuído em uma rede com múltiplos roteadores.

\textbf{Objetivos Específicos:}

\begin{itemize}[label=\checkbox]
    \item Implementar comunicação TCP para transferência confiável de dados
    \item Implementar comunicação UDP para monitoramento em tempo real
    \item Criar interface gráfica Swing intuitiva para o cliente
    \item Configurar rede com 3 roteadores no Cisco Packet Tracer
    \item Implementar port forwarding/NAT para roteamento
    \item Utilizar multithreading para processamento paralelo
    \item Validar comunicação usando Wireshark
\end{itemize}

\subsection{Justificativa do Contexto}

A escolha de um sistema de processamento distribuído de algoritmos permite:

\begin{enumerate}
    \item \textbf{Relevância Prática}: Simula ambientes reais de cloud computing e clusters de processamento
    \item \textbf{Complexidade Adequada}: Envolve transferência de dados complexos, não apenas mensagens simples
    \item \textbf{Multidisciplinar}: Integra conhecimentos de diversas áreas da Ciência da Computação
    \item \textbf{Demonstração Clara}: Permite visualizar diferenças entre TCP (confiável) e UDP (rápido)
\end{enumerate}

\section{FUNDAMENTAÇÃO TEÓRICA}

\subsection{Protocolo TCP (Transmission Control Protocol)}

\textbf{Características:}
\begin{itemize}
    \item Orientado à conexão (three-way handshake)
    \item Confiável (retransmissão de pacotes perdidos)
    \item Ordenado (garante ordem de chegada)
    \item Controle de fluxo e congestionamento
\end{itemize}

\textbf{Uso no Projeto:}
\begin{itemize}
    \item Porta: \textbf{5000}
    \item Função: Envio de tarefas computacionais e recebimento de resultados
    \item Justificativa: A integridade dos dados é crítica para tarefas computacionais
\end{itemize}

\subsection{Protocolo UDP (User Datagram Protocol)}

\textbf{Características:}
\begin{itemize}
    \item Sem conexão (connectionless)
    \item Não confiável (sem garantia de entrega)
    \item Sem ordenação
    \item Baixa latência
\end{itemize}

\textbf{Uso no Projeto:}
\begin{itemize}
    \item Porta: \textbf{5001}
    \item Função: Monitoramento de métricas do servidor em tempo real
    \item Justificativa: Perda ocasional de pacotes de métricas é aceitável
\end{itemize}

\subsection{Multithreading}

\textbf{Thread Pools Implementados:}

\begin{enumerate}
    \item \textbf{Thread Pool de Tarefas} (ExecutorService com 4 threads fixas)
    \begin{itemize}
        \item Processa tarefas computacionais simultaneamente
        \item Evita criação excessiva de threads
    \end{itemize}
    
    \item \textbf{Thread Pool de Rede} (ExecutorService com threads dinâmicas)
    \begin{itemize}
        \item Gerencia conexões de rede
        \item Escala conforme demanda
    \end{itemize}
    
    \item \textbf{SwingWorker}
    \begin{itemize}
        \item Evita bloqueio da interface gráfica
        \item Atualiza UI de forma segura
    \end{itemize}
\end{enumerate}

\subsection{NAT e Port Forwarding}

\textbf{Network Address Translation (NAT):}
\begin{itemize}
    \item Permite comunicação entre redes privadas diferentes
    \item Traduz endereços IP na camada de rede
\end{itemize}

\textbf{Port Forwarding:}
\begin{itemize}
    \item Redireciona tráfego de uma porta específica
    \item Essencial para acessar servidores em redes internas
\end{itemize}

\section{ARQUITETURA DO SISTEMA}

\subsection{Visão Geral}

\begin{figure}[H]
    \centering
    \begin{verbatim}
________________                                     ________________
|              |        TCP (porta 5000)             |              |
|   PC1        |------------------------------------>|     PC2      |
|  (Cliente)   |        Tarefas e Arquivos           |  (Servidor)  |
|              |                                     |              |
| 192.168.1.10 |<------------------------------------|  10.0.0.100  |
|              |        Resultados                   |              |
|              |                                     |              |
|              |        UDP (porta 5001)             |              |
|              |------------------------------------>|              |
|              |        Métricas em tempo real       |              |
|______________|                                     |______________|
       |                                                      ^
       |                                                      |
       V                                                      |
  _________          _________          _________             |
  |   R1  |--------> |   R2  |--------> |   R3  |_____________|
  |192.168|          | 172.16|          |  10.0 |
  |/16    |          |  /12  |          |  /8   |
  |_______|          |_______|          |_______|
    \end{verbatim}
    \caption{Arquitetura geral do sistema}
    \label{fig:arquitetura}
\end{figure}

\subsection{Componentes do Sistema}

\subsubsection{ProcessingClient (PC1)}
\begin{itemize}
    \item Interface gráfica Swing
    \item Conexão TCP para envio de tarefas
    \item Cliente UDP para monitoramento
    \item 5 tipos de tarefas disponíveis
\end{itemize}

\subsubsection{ProcessingServer (PC2)}
\begin{itemize}
    \item Servidor TCP (porta 5000)
    \item Servidor UDP (porta 5001)
    \item Pool de 4 threads para processamento
    \item Autenticação com SHA-256
\end{itemize}

\subsection{Fluxo de Comunicação}

\subsubsection{Fluxo TCP - Execução de Tarefa:}

\begin{enumerate}
    \item Cliente estabelece conexão TCP com servidor
    \item Cliente envia credenciais (AUTH\_KEY + PASSWORD)
    \item Servidor valida credenciais
    \item Cliente envia tipo de tarefa + dados
    \item Servidor processa tarefa em thread separada
    \item Servidor envia resultado de volta
    \item Conexão é encerrada
\end{enumerate}

\subsubsection{Fluxo UDP - Monitoramento:}

\begin{enumerate}
    \item Cliente envia requisição ``STATUS'' via UDP
    \item Servidor responde com JSON de métricas
    \item Cliente atualiza interface
    \item Repete a cada 2 segundos
\end{enumerate}

\section{IMPLEMENTAÇÃO}

\subsection{Tecnologias Utilizadas}

\begin{table}[H]
\centering
\begin{tabular}{@{}lll@{}}
\toprule
\textbf{Tecnologia} & \textbf{Versão} & \textbf{Uso} \\ \midrule
Java & 8+ & Linguagem principal \\
Java Swing & - & Interface gráfica \\
Java Sockets & - & Comunicação TCP/UDP \\
ExecutorService & - & Multithreading \\
Cisco Packet Tracer & 8.x & Simulação de rede \\
Wireshark & 4.x & Análise de pacotes \\ \bottomrule
\end{tabular}
\caption{Tecnologias utilizadas no projeto}
\label{tab:tecnologias}
\end{table}

\subsection{Estrutura de Classes}

\begin{lstlisting}[language=Java, caption=Estrutura do projeto]
gui/
├── ProcessingServer.java    # Servidor principal
│   ├── Servidor TCP (porta 5000)
│   ├── Servidor UDP (porta 5001)
│   ├── Algoritmos de ordenação
│   ├── Algoritmos de busca
│   └── Processamento de matrizes
│
├── ProcessingClient.java    # Cliente com interface Swing
│   ├── Conexão TCP
│   ├── Monitoramento UDP
│   └── Interface gráfica
│
├── Config.java              # Gerenciador de configurações
└── config.properties        # Arquivo de credenciais
\end{lstlisting}

\subsection{Algoritmos Implementados}

\subsubsection{Ordenação:}
\begin{itemize}
    \item \textbf{Bubble Sort}: $O(n^2)$ - Simples, educacional
    \item \textbf{Quick Sort}: $O(n \log n)$ médio - Rápido, recursivo
    \item \textbf{Merge Sort}: $O(n \log n)$ - Estável, divide-e-conquista
\end{itemize}

\subsubsection{Busca:}
\begin{itemize}
    \item \textbf{Linear}: $O(n)$ - Simples, não requer ordenação
    \item \textbf{Binária}: $O(\log n)$ - Rápida, requer array ordenado
\end{itemize}

\subsubsection{Outros:}
\begin{itemize}
    \item \textbf{Multiplicação de Matrizes}: $O(n^3)$
    \item \textbf{Crivo de Eratóstenes}: $O(n \log \log n)$
\end{itemize}

\subsection{Segurança}

\textbf{Autenticação:}
\begin{lstlisting}[language=Java, caption=Configuração de autenticação]
String AUTH_KEY = "yourKey";
String PASSWORD_HASH = sha256("yourPassword");
\end{lstlisting}

\textbf{Validação:}
\begin{itemize}
    \item Todas as entradas são validadas
    \item Nomes de arquivo são sanitizados
    \item Timeouts em conexões
\end{itemize}

\section{CONFIGURAÇÃO DE REDE}

\subsection{Topologia}

% TODO: INSERIR SCREENSHOT: Topologia completa no Packet Tracer
% \begin{figure}[H]
%     \centering
%     \includegraphics[width=0.8\textwidth]{screenshots/topology.png}
%     \caption{Topologia completa no Cisco Packet Tracer}
%     \label{fig:topology}
% \end{figure}

\textbf{Descrição:}
\begin{itemize}
    \item 2 PCs (cliente e servidor)
    \item 3 Roteadores (R1, R2, R3)
    \item 3 redes diferentes (/16, /12, /8)
\end{itemize}

\subsection{Tabela de Endereçamento}

\begin{table}[H]
\centering
\begin{tabular}{@{}lllll@{}}
\toprule
\textbf{Dispositivo} & \textbf{Interface} & \textbf{Endereço IP} & \textbf{Máscara} & \textbf{Gateway} \\ \midrule
PC1 & Eth0 & 192.168.1.10 & 255.255.0.0 & 192.168.0.1 \\
R1 & Fa0/0 (LAN) & 192.168.0.1 & 255.255.0.0 & - \\
R1 & Fa0/1 (WAN) & 172.16.0.2 & 255.240.0.0 & - \\
R2 & Fa0/0 (WAN) & 172.16.0.1 & 255.240.0.0 & - \\
R2 & Fa0/1 (LAN) & 10.0.0.2 & 255.0.0.0 & - \\
R3 & Fa0/0 (WAN) & 10.0.0.1 & 255.0.0.0 & - \\
R3 & Fa0/1 (LAN) & 10.0.0.1 & 255.0.0.0 & - \\
PC2 & Eth0 & 10.0.0.100 & 255.0.0.0 & 10.0.0.1 \\ \bottomrule
\end{tabular}
\caption{Tabela de endereçamento da rede}
\label{tab:enderecos}
\end{table}

\subsection{Configuração dos Roteadores}

% TODO: INSERIR SCREENSHOTS: Configuração CLI de R1, R2, R3

\textbf{Comandos Essenciais:}

\textbf{R1:}
\begin{lstlisting}[language=bash, caption=Configuração do Roteador R1]
ip nat inside source static tcp 172.16.0.1 5000 192.168.0.1 5000
ip route 10.0.0.0 255.0.0.0 172.16.0.1
\end{lstlisting}

\textbf{R2:}
\begin{lstlisting}[language=bash, caption=Configuração do Roteador R2]
ip nat inside source static tcp 10.0.0.100 5000 172.16.0.2 5000
ip route 192.168.0.0 255.255.0.0 172.16.0.2
ip route 10.0.0.0 255.0.0.0 10.0.0.1
\end{lstlisting}

\textbf{R3:}
\begin{lstlisting}[language=bash, caption=Configuração do Roteador R3]
ip nat inside source static tcp 10.0.0.100 5000 10.0.0.1 5000
ip route 0.0.0.0 0.0.0.0 10.0.0.2
\end{lstlisting}

\subsection{Teste de Conectividade}

% TODO: INSERIR SCREENSHOT: ping de PC1 para PC2

\textbf{Comando:}
\begin{lstlisting}[language=bash, caption=Teste de ping]
PC1> ping 10.0.0.100
\end{lstlisting}

\textbf{Resultado Esperado:}
\begin{lstlisting}[language=bash]
Reply from 10.0.0.100: bytes=32 time=20ms TTL=125
Reply from 10.0.0.100: bytes=32 time=18ms TTL=125
Reply from 10.0.0.100: bytes=32 time=19ms TTL=125
Reply from 10.0.0.100: bytes=32 time=21ms TTL=125
\end{lstlisting}

\section{TELAS DA APLICAÇÃO}

\subsection{Servidor (ProcessingServer)}

% TODO: INSERIR SCREENSHOT: Tela principal do servidor
% \begin{figure}[H]
%     \centering
%     \includegraphics[width=0.8\textwidth]{screenshots/server.png}
%     \caption{Tela principal do servidor}
%     \label{fig:server}
% \end{figure}

\textbf{Funcionalidades Visíveis:}
\begin{itemize}
    \item Botões de iniciar/parar TCP e UDP
    \item Lista de tarefas processadas com timestamps
    \item Métricas do sistema (uptime, tarefas, threads, memória)
    \item Log de eventos em tempo real
    \item Status de conexão
\end{itemize}

\subsection{Cliente (ProcessingClient)}

% TODO: INSERIR SCREENSHOT: Tela principal do cliente
% \begin{figure}[H]
%     \centering
%     \includegraphics[width=0.8\textwidth]{screenshots/client.png}
%     \caption{Tela principal do cliente}
%     \label{fig:client}
% \end{figure}

\textbf{Funcionalidades Visíveis:}
\begin{itemize}
    \item Configuração de conexão (host, portas TCP/UDP)
    \item Seleção de tipo de tarefa
    \item Configuração específica para cada tarefa
    \item Área de resultados
    \item Métricas do servidor via UDP
    \item Barra de progresso
\end{itemize}

\subsection{Execução de Tarefas}

% TODO: INSERIR SCREENSHOTS de cada tarefa
% \begin{figure}[H]
%     \centering
%     \begin{minipage}{0.45\textwidth}
%         \centering
%         \includegraphics[width=\textwidth]{screenshots/sort.png}
%         \caption{Execução de SORT}
%     \end{minipage}\hfill
%     \begin{minipage}{0.45\textwidth}
%         \centering
%         \includegraphics[width=\textwidth]{screenshots/search.png}
%         \caption{Execução de SEARCH}
%     \end{minipage}
% \end{figure}

\section{ANÁLISE NO WIRESHARK}

\subsection{Captura TCP - Porta 5000}

% TODO: INSERIR SCREENSHOT: Wireshark mostrando tráfego TCP
% \begin{figure}[H]
%     \centering
%     \includegraphics[width=\textwidth]{screenshots/wireshark_tcp.png}
%     \caption{Captura de tráfego TCP no Wireshark}
%     \label{fig:wireshark_tcp}
% \end{figure}

\textbf{Filtro Utilizado:}
\begin{lstlisting}[language=bash]
tcp.port == 5000
\end{lstlisting}

\textbf{Análise:}

\subsubsection{Three-Way Handshake:}
\begin{enumerate}
    \item \textbf{SYN}: Cliente $\rightarrow$ Servidor (seq=0)
    \item \textbf{SYN-ACK}: Servidor $\rightarrow$ Cliente (seq=0, ack=1)
    \item \textbf{ACK}: Cliente $\rightarrow$ Servidor (ack=1)
\end{enumerate}

% TODO: INSERIR SCREENSHOT: Detalhes do handshake

\subsubsection{Transferência de Dados:}
\begin{itemize}
    \item \textbf{PSH, ACK}: Pacotes com dados da tarefa
    \item \textbf{ACK}: Confirmações de recebimento
    \item \textbf{Window Size}: Controle de fluxo visível
\end{itemize}

% TODO: INSERIR SCREENSHOT: Pacotes de dados

\subsubsection{Finalização da Conexão:}
\begin{enumerate}
    \item \textbf{FIN, ACK}: Cliente fecha conexão
    \item \textbf{FIN, ACK}: Servidor confirma
    \item \textbf{ACK}: Cliente confirma
\end{enumerate}

\subsection{Captura UDP - Porta 5001}

% TODO: INSERIR SCREENSHOT: Wireshark mostrando tráfego UDP
% \begin{figure}[H]
%     \centering
%     \includegraphics[width=\textwidth]{screenshots/wireshark_udp.png}
%     \caption{Captura de tráfego UDP no Wireshark}
%     \label{fig:wireshark_udp}
% \end{figure}

\textbf{Filtro Utilizado:}
\begin{lstlisting}[language=bash]
udp.port == 5001
\end{lstlisting}

\textbf{Análise:}

\subsubsection{Requisições de Status:}
\begin{itemize}
    \item Sem handshake
    \item Pacotes diretos de requisição
    \item Tamanho pequeno (< 100 bytes)
\end{itemize}

\subsubsection{Respostas com Métricas:}
\begin{itemize}
    \item JSON com métricas do servidor
    \item Sem confirmação de recebimento
    \item Possível perda de pacotes (aceitável)
\end{itemize}

% TODO: INSERIR SCREENSHOT: Conteúdo do pacote UDP com JSON

\subsection{Comparação TCP vs UDP}

\begin{table}[H]
\centering
\begin{tabular}{@{}lll@{}}
\toprule
\textbf{Característica} & \textbf{TCP (5000)} & \textbf{UDP (5001)} \\ \midrule
Handshake & Sim (3-way) & Não \\
Confirmação & ACKs & Nenhuma \\
Overhead & Alto ($\sim$40 bytes) & Baixo ($\sim$8 bytes) \\
Latência & Maior & Menor \\
Confiabilidade & Garantida & Não garantida \\
Uso no Projeto & Tarefas críticas & Monitoramento \\ \bottomrule
\end{tabular}
\caption{Comparação entre TCP e UDP}
\label{tab:tcp_udp}
\end{table}

\subsection{Análise de Roteamento}

% TODO: INSERIR SCREENSHOT: Traceroute PC1 → PC2

\textbf{Caminho dos Pacotes:}
\begin{enumerate}
    \item 192.168.1.10 (PC1)
    \item 192.168.0.1 (R1)
    \item 172.16.0.1 (R2)
    \item 10.0.0.1 (R3)
    \item 10.0.0.100 (PC2)
\end{enumerate}

\section{RESULTADOS E DISCUSSÃO}

\subsection{Performance dos Algoritmos}

\subsubsection{Teste 1: Ordenação de 10.000 elementos}

\begin{table}[H]
\centering
\begin{tabular}{@{}lll@{}}
\toprule
\textbf{Algoritmo} & \textbf{Tempo (ms)} & \textbf{Complexidade} \\ \midrule
Bubble Sort & 245.3 & $O(n^2)$ \\
Quick Sort & 3.8 & $O(n \log n)$ \\
Merge Sort & 4.2 & $O(n \log n)$ \\ \bottomrule
\end{tabular}
\caption{Performance dos algoritmos de ordenação}
\label{tab:sort_perf}
\end{table}

\textbf{Observação}: Quick Sort e Merge Sort são significativamente mais rápidos para grandes volumes de dados.

\subsubsection{Teste 2: Busca em array de 100.000 elementos}

\begin{table}[H]
\centering
\begin{tabular}{@{}lll@{}}
\toprule
\textbf{Algoritmo} & \textbf{Tempo ($\mu$s)} & \textbf{Resultado} \\ \midrule
Busca Linear & 1250.5 & Encontrado \\
Busca Binária & 2.3 & Encontrado \\ \bottomrule
\end{tabular}
\caption{Performance dos algoritmos de busca}
\label{tab:search_perf}
\end{table}

\textbf{Observação}: Busca binária é $\sim$500x mais rápida, mas requer array ordenado.

\subsection{Métricas do Sistema}

\textbf{Durante 10 minutos de operação:}
\begin{itemize}
    \item Tarefas processadas: 47
    \item Threads ativas (pico): 4
    \item Memória utilizada (média): 85 MB
    \item Uptime: 00:10:32
\end{itemize}

\subsection{Latência de Rede}

\textbf{Medições de RTT (Round-Trip Time):}
\begin{itemize}
    \item Média: 19.5 ms
    \item Mínima: 15 ms
    \item Máxima: 28 ms
\end{itemize}

\textbf{Impacto dos 3 roteadores:}
\begin{itemize}
    \item Cada hop adiciona $\sim$5-7 ms de latência
\end{itemize}

\section{DESAFIOS E SOLUÇÕES}

\subsection{Problema: Port Forwarding Complexo}

\textbf{Desafio}: Configurar NAT corretamente em 3 roteadores.

\textbf{Solução}:
\begin{itemize}
    \item Mapeamento cuidadoso de portas em cada roteador
    \item Testes incrementais (primeiro 1 roteador, depois 2, depois 3)
    \item Uso de \texttt{show ip nat translations} para debug
\end{itemize}

\subsection{Problema: Sincronização de Threads}

\textbf{Desafio}: Atualizar UI Swing de threads secundárias.

\textbf{Solução}:
\begin{itemize}
    \item Uso de \texttt{SwingUtilities.invokeLater()} para todas as atualizações de UI
    \item SwingWorker para tarefas longas
\end{itemize}

\subsection{Problema: Perda de Pacotes UDP}

\textbf{Desafio}: Métricas ocasionalmente não apareciam.

\textbf{Solução}:
\begin{itemize}
    \item Implementação de timeout no socket UDP
    \item Retry automático após 2 segundos
    \item Aceitação de perda ocasional (natureza do UDP)
\end{itemize}

\section{CONCLUSÃO}

\subsection{Objetivos Alcançados}

\begin{itemize}[label=\checkbox]
    \item \textbf{TCP para Comunicação Confiável}: Implementado com sucesso para envio de tarefas computacionais
    \item \textbf{UDP para Monitoramento}: Funcional para métricas em tempo real
    \item \textbf{Interface Gráfica Swing}: Interface intuitiva e responsiva
    \item \textbf{Rede com 3 Roteadores}: Configurada corretamente no Packet Tracer
    \item \textbf{Port Forwarding/NAT}: Implementado em todos os roteadores
    \item \textbf{Multithreading}: Thread pools funcionando eficientemente
    \item \textbf{Validação com Wireshark}: Tráfego TCP e UDP analisado e compreendido
\end{itemize}

\subsection{Aprendizados}

\begin{enumerate}
    \item \textbf{Diferenças práticas entre TCP e UDP}: Entendimento claro de quando usar cada protocolo
    \item \textbf{Complexidade de roteamento}: NAT e port forwarding em múltiplos roteadores requer planejamento cuidadoso
    \item \textbf{Importância de multithreading}: Essencial para aplicações de rede não bloqueantes
    \item \textbf{Análise de pacotes}: Wireshark é fundamental para debug de aplicações de rede
    \item \textbf{Design de sistemas distribuídos}: Separação de responsabilidades entre cliente e servidor
\end{enumerate}

\subsection{Trabalhos Futuros}

\begin{itemize}
    \item Implementar balanceamento de carga com múltiplos servidores
    \item Adicionar criptografia TLS/SSL para TCP
    \item Dashboard web para visualização de métricas
    \item Deployar em ambiente de cloud real (AWS, Azure)
    \item Adicionar mais algoritmos (FFT, DFT, compressão)
\end{itemize}

\subsection{Considerações Finais}

Este projeto demonstrou com sucesso a aplicação prática de conceitos de redes de computadores, combinando teoria e prática. A integração com outras disciplinas (Arquitetura, Sistemas Operacionais, Algoritmos) enriqueceu o aprendizado e mostrou a interconexão dos conhecimentos em Ciência da Computação.

A escolha de um contexto realista (processamento distribuído) permitiu compreender aplicações práticas dos protocolos TCP e UDP, além de desafios reais de sistemas distribuídos.

\begin{thebibliography}{9}

\bibitem{kurose2013}
KUROSE, J. F.; ROSS, K. W.
\textit{Redes de Computadores e a Internet: Uma Abordagem Top-Down}.
6ª ed. Pearson, 2013.

\bibitem{tanenbaum2011}
TANENBAUM, A. S.; WETHERALL, D.
\textit{Redes de Computadores}.
5ª ed. Pearson, 2011.

\bibitem{oracle}
ORACLE.
\textit{Java Network Programming}.
Disponível em: \url{https://docs.oracle.com/javase/tutorial/networking/}

\bibitem{cisco}
CISCO.
\textit{Cisco Packet Tracer Documentation}.
Disponível em: \url{https://www.netacad.com/}

\bibitem{wireshark}
WIRESHARK.
\textit{User's Guide}.
Disponível em: \url{https://www.wireshark.org/docs/}

\bibitem{cormen2012}
CORMEN, T. H. et al.
\textit{Algoritmos: Teoria e Prática}.
3ª ed. Campus, 2012.

\end{thebibliography}

\end{document}
